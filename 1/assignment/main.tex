\documentclass[onecolumn, draftclsnofoot,10pt, compsoc]{IEEEtran}
\usepackage{url}
\usepackage{setspace}
% Code listings: using either pygments or listings package. Use color for syntax highlighting.
%\usepackage{minted}
% allowing using urls
%\usepackage{hyperref}
% Captions: required on all figures, must be meaningful
% to add images, add into figures/ folder
%\usepackage[pdftex]{graphicx}
%\graphicspath{ {figures/} }
% citation
%\usepackage{cite}
% math
\usepackage{amsmath}
% algorithm
\usepackage{algorithmic}
% alignment
\usepackage{array}
% for languages
\usepackage{listings}
\usepackage{geometry}
\geometry{textheight=9.5in, textwidth=7in}

% 1. Fill in these details
\def \CapstoneTeamNumber{		5}
\def \GroupMemberOne{			Alexandar Hull \textit{(hullale)}}
\def \GroupMemberTwo{			Allison Sladek \textit{(sladeka)}}
\def \GroupMemberThree{			Anish Asrani \textit{(asrania)}}
\def \GroupMemberFour{			Mazen Alotaibi \textit{(alotaima)}}

% 2. Uncomment the appropriate line below so that the document type works
\def \DocType{ Assignment 1}
			
\newcommand{\NameSigPair}[1]{\par
\makebox[2.75in][r]{#1} \hfil 	\makebox[3.25in]{\makebox[2.25in]{\hrulefill} \hfill		\makebox[.75in]{\hrulefill}}
\par\vspace{-12pt} \textit{\tiny\noindent
\makebox[2.75in]{} \hfil		\makebox[3.25in]{\makebox[2.25in][r]{Signature} \hfill	\makebox[.75in][r]{Date}}}}
% 3. If the document is not to be signed, uncomment the RENEWcommand below
\renewcommand{\NameSigPair}[1]{#1}

%%%%%%%%%%%%%%%%%%%%%%%%%%%%%%%%%%%%%%%
\begin{document}
\begin{titlepage}
    \pagenumbering{gobble}
    \begin{singlespace}
 %   	\includegraphics[height=4cm]{coe_v_spot1}
        \hfill 
        % 4. If you have a logo, use this includegraphics command to put it on the coversheet.
        %\includegraphics[height=4cm]{CompanyLogo}   
        \par\vspace{.2in}
        \centering
        \scshape{
            \huge \DocType \par
            {\large\today}\par
            \vspace{.5in}
           \vfill
            \vspace{5pt}
            {\large Prepared by }\par
            Group\CapstoneTeamNumber\par
            \vspace{5pt}
            {\Large
                \NameSigPair{\GroupMemberOne}\par
                \NameSigPair{\GroupMemberTwo}\par
                \NameSigPair{\GroupMemberThree}\par
                \NameSigPair{\GroupMemberFour}\par
            }
            \vspace{20pt}
        }
        \begin{abstract}
        This assignment served as an introduction to using the Linux kernel and writing code with parallelism. We explored using the emulator and the toolkit with the kernel, paving the way for more exciting kernel applications in the future.
        With concurrency, we wrote a basic program to demonstrate scheduling and managing threads. Grasping the inner-workings of this is vital to understand how an operating system functions and showcases things that are not directly visible to the average user. 
        
        
        \end{abstract}     
    \end{singlespace}
\end{titlepage}
\newpage
\pagenumbering{arabic}
\tableofcontents
% 7. uncomment this (if applicable). Consider adding a page break.
%\listoffigures
%\listoftables
\clearpage

% 8. now you write!
\section{Log of Commands}

\begin{lstlisting}[language=bash]
    -bash-4.2$ ssh os2
    -bash-4.2$ mkdir /scratch/fall2018/group5
    -bash-4.2$ cd /scratch/fall2018/group5
    -bash-4.2$ git clone git://git.yoctoproject.org/linux-yocto-3.19
    -bash-4.2$ git checkout tags/v3.19.2
    -bash-4.2$ cp /scratch/files/core-image-lsb-sdk-qemux86.ext4 .
    -bash-4.2$ cp /scratch/files/bzImage-qemux86.bin .
    -bash-4.2$ source /scratch/files/environment-setup-i586-poky-linux
    -bash-4.2$ qemu-system-i386 -gdb tcp::5505 -S -nographic -kernel
               bzImage-qemux86.bin -drive file=core-image-lsb-sdk-qemux86.ext4,
               if=virtio -enable-kvm -net none -usb -localtime --no-reboot 
               --append "root=/dev/vda rw console=ttyS0 debug"
\end{lstlisting}
In another terminal window: (Alternatively could use screen)
\begin{lstlisting}[language=bash]          
    -bash-4.2$ gbd
    (gdb)  target remote tcp:localhost:5505
    (gdb)  c
    qemux86 login: root
    (gdb)  quit
     -bash-4.2$ fuser -k 5505/tcp
\end{lstlisting}


\section{Qemu Command-line}

\begin{itemize}
    \item \textbf{-gdb tcp::5505}= Wait for a gdb connection on TCP port 5505.
    \item \textbf{-S}= Do not start CPU on startup (user will have to manually type 'c').
    \item \textbf{-nographic}= Causes qemu to be only a command line to increase speed and avoid unnecessary graphics.
    \item \textbf{-kernel bzImage-qemux86.bin}= Specify the kernel image path.
    \item \textbf{-drive file=core-image-lsb-sdk-qemux86.ext4,if=virtio}= Define a new drive with the path to the image. Connect the drive with virtio interface.
    \item \textbf{-enable-kvm}= Enable KVM full virtualization support. KVM=Kernel-based Virtual Machine.
    \item \textbf{-net none}= Avoid creating a network interface card.
    \item \textbf{-usb}= Enable the USB driver.
    \item \textbf{-localtime}= Sets the time to the current UTC local time.
    \item \textbf{--no-reboot}= Exit instead of rebooting.
    \item \textbf{--append "root=/dev/vda rw console=ttyS0 debug"}= Use "root=/dev/vda rw console=ttyS0 debug" as kernel command line.
\end{itemize}

\section{Concurrency}
\subsection{What do you think the main point of this assignment is?}
The main point of this assignment was to introduce parallelism and gain some experience in writing threaded programs. 
Concurrency is a good way to demonstrate some of the functions a kernel regularly performs, like scheduling and running processes. It gives us a better understanding about how resources are managed by a computer. 
Knowing how to write this kind of program will likely become important to understanding some of the topics we'll be covering in class.

\subsection{How did you personally approach the problem? Design decisions, algorithm, etc.}
We chose to logically split the tasks into individual c and header files. For example, we have a file for the producer, consumer, random, and one for the tasks. Each of these files has only 1-3 functions. We felt that splitting them up into individual files would be easier to understand and would make the program feel more organized. 

We decided to store the tasks in an array of structs. This array is statically allocated to a size of BUFFER\_SIZE which is globally defined. This array is essentially a queue, except it wraps around. There are two integers, bufferHead and bufferTail. bufferHead indicates the next available spot in the array. bufferTail is the index of oldest task, which should be removed next. We used the modulo operator to deal with the wrapping. This form was chosen because it was easy to implement, and didn't require any dynamic memory allocation or raw pointers. Alternatively, we could have kept the array sorted after every update instead of wrapping around, but this would slow down the algorithm. 


\subsection{How did you ensure your solution was correct? Testing details, for instance.}
We started to test smaller sections of our code to ensure they were spewing values within the range that we expected and then expanded from there. There were multiple runs to verify that we weren't just getting the results by chance.

To test behavior when the buffer is full or empty, we added an extra thread for 2 producers or consumers. This way the producers would always be faster than the consumer so the buffer would eventually fill up. We did the same by adding 2 consumers to make sure the buffer would be empty. 


\subsection{What did you learn?}
This exercise gave us more practice with parallel programming. It also gave us a better understanding of how tasks may be scheduled and managed. 

\section{Version Control Log}
\begin{center}
  \begin{tabular}{ | l | c | c | c |}
    \hline
    Description & Date & Name & Commit \\ \hline
    Added readme & Oct 3, 2018 & Anish & 5c1ccfe5d9dfbf11d9aab95deeb933a385fa8dda\\ \hline
    Added kernel source files & Oct 3, 2018 & Allison & b62e2eb6820bdaedb8d686509ad6c77ba8e2ec7c\\ \hline
    Save readme to local branch & Oct 3, 2018 & Allison & 7a72a85dbc5390cb96970549ba07c183e62c68f3\\ \hline
    Merge pull request \#1 from ASladek/master & Oct 3, 2018 & Alexander & 9e40db94a87c1ebb79d01ed62aa2ef9d553fc525\\ \hline
    Added skeleton code & Oct 9, 2018 & Alexander & 8b148072d2d470e582f7addf05da7ceceb3c3287 \\ \hline
    Added some task code & Oct 10, 2018 & Alexander & eff657f6d5a91b8acfa01d9517920f4502210bbc \\ \hline
    Added Empty/Full checks. Now call & Oct 10, 2018 & Alexander & de28c3b10e5de0339023b78b9c11d517524b5b6d \\ 
    produce/consume only & ~ & ~ & ~\\ \hline    
    Merge pull request \#2 from hullale/master & Oct 13, 2018 & Anish & e481223aa1106e40422e12bb6bff3b7813efa213 \\ \hline
    Added threading, rand, and sleeps & Oct 14, 2018 & Alexander & e2fde6fbbbd353f3bea548f602549f94ce658338 \\ \hline
    Merge pull request \#3 from hullale/master & Oct 14, 2018 & Mazen & bd6031d2cfac945683828fa08bc2ac5b6c99ed8d \\ \hline
    support for rdrand & Oct 14, 2018 & Alexander & 728f484dd413b72aa14832ea85e7441db31b1c44 \\ \hline
    Merge pull request \#4 from hullale/master & Oct 14, 2018 & Alexander & 30e04fd582af5c69611901ad3095ff4d7bcb0d0e \\ \hline
    "Added 32 buffer size, rdrand, print & Oct 14, 2018 & Mazen & fd8ac0af00f28a2917ce1081abfa86fb184ee842 \\ 
    formatting"& ~ & ~ & ~\\ \hline
    Remove comment & Oct 14, 2018 & Mazen & 6f7eb6fe3877ee906dcee45c8b86376e213fde24 \\ \hline
    assignment folder & Oct 14, 2018 & Alexander & 6e8966180be02ec614b7fa526ca6fce6e82e1e8c \\ \hline
  \end{tabular}
\end{center}

\section{Work Log}
\subsection{October 3rd, 2018}
The group convened to read the assignment description, set up a repository, and coordinate tasks. 
We chose Github because we're all familiar with it and can set our repository to private. 
The initial clone of the kernel took longer than expected as we ran into trouble with git modules. 
All group members were then granted access to the group5 scratch folder.

\subsection{October 7th, 2018}
We walked through the process of running the kernel together, making sure each group member was able to run the commands to test the emulator and the toolchain, completing the "Getting Acquainted" portion of the assignment.
The files for the write-up were initialized and we began filling in sections.

\subsection{October 8th, 2018}
Today there was some minor report reformatting.

\subsection{October 9th, 2018}
Alex added skeleton code for the concurrency section of the assignment.

\subsection{October 10th, 2018}
Alex updated the concurrency files, adding tasks and calling produce() and consume() without any concurrency yet.

\subsection{October 14th, 2018}
The group convened to pair-program for the concurrency part where we collectively worked on a shared screen. Most of the testing was done on the engineering machines.

\subsection{October 15th, 2018}
The write-up was completed.% WILL be completed


%\bibliographystyle{IEEEtran}  
%\bibliography{sources}  

\end{document}
